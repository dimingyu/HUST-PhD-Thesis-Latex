
%%% Local Variables:
%%% mode: latex
%%% TeX-master: t
%%% End:

\ctitle{基于多模态的时序网络入侵检测模型研究}

\xuehao{M202171758} 
% \xuehao{D201880970} 
\schoolcode{10487}
\csubjectname{网络空间安全} 
\cauthorname{狄明宇}
\csupervisorname{{吴俊军}} 
% \cauthorname{***}
% \csupervisorname{***} 
\csupervisortitle{副教授}
\defencedate{2024~年~5~月~XX~日} \grantdate{}
\chair{}%
\firstreviewer{} \secondreviewer{} \thirdreviewer{}


\etitle{Research on Multimodal Time-Series Network Intrusion Detection Model}
\edegree{Master of Engineering}
\esubject{Cyberspace Security}
\eauthor{Di Mingyu}
\esupervisor{Prof. Wu Junjun}
% \eauthor{***** *****}
% \esupervisor{Prof. *** *****}
\edate{May, 2024}

\dctab{\begin{tabular}{|P{0.9cm}|P{1.8cm}|P{1.8cm}|P{5.4cm}|}
    \hline
    &{\hei\textbf{姓名}}&{ \hei\textbf{职称}}&{\hei \textbf{单位}}\\
    \hline
    主席&X\hspace{1em}X&教授&武汉大学~信息管理学院\\
    \hline
    \multirow{4}{*}{委员}&X\hspace{1em}X&教授&武汉大学~信息管理学院\\
    \cline{2-4}
    &XXX&教授&华中科技大学~管理学院\\
    \cline{2-4}
    &XXX&教授&华中科技大学~管理学院\\
    \cline{2-4}
    &X\hspace{1em}X&教授&华中科技大学~管理学院\\
    \hline
\end{tabular}
}


%定义中英文摘要和关键字
\cabstract{
网络已经融入社会生活方方面面,网络安全意义重大,网络入侵检测技术随着信息化进步不断发展,网络入侵检测是网络安全的前提,其性能对网络安全运行有着重要的意义。在使用机器学习模型设计网络入侵检测方法时常使用“公开数据集”进行训练模型,实际应用时由于“公开数据集”与实际网络环境存在差异,使用“公开数据集”训练的模型在实际应用时存在性能存在下降的问题。

针对“公开数据集”训练的模型在实际网络环境下应用时性能下降的问题,本论文提出了基于Transformer 架构的多模态时序网络入侵检测模型。

论文主要研究内容由三部分组成:第一部分主要设计了一种新的多模态通用流量编码器,其核心在于新的数据嵌入方法以及新的位置编码方法,前者可将数值与文本嵌入到同一空间中,后者使用数据集描述的文本信息生成位置编码,该编码器使用Transformer 对信息进行有效融合;第二部分利用Transformer 捕获流量的时序特征,进而提高模型对网络流量的识别能力;第三部分,针对不同数据集之间数据分布不同的特点,设计了一种新的基于通用模型的迁移学习方法,减少迁移过程中的信息损失,进而实现更高的分类准确率。

实验结果表明,论文提出的多模态通用编码器可提升时序模型在不均衡数据集上的性能,其中全Transformer 架构性能最为突出,在CIC-IDS2017 数据集上达99.83\% 的准确率。在迁移学习中仅需万分之二的数据就能避免模型过拟合,在使用10\% 的数据时,准确率为99.82\%,高于传统的基于模型的迁移学习方式。这也证明论文提出的“完全基于Transformer 架构时序多模态的入侵检测模型”在网络入侵检测方面具有一定的先进性,对解决使用“公开数据集”训练的模型在实际应用时网络入侵检测性能下降问题有较好的效果。
}

\ckeywords{网络入侵检测;多模态;时序模型;Transformer;迁移学习}

\eabstract{
    XXX
}

\ekeywords{Network Intrusion Detection; Multimodal; Time-series model; Transformer; Transfer Learning
}
