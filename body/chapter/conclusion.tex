%%% mode: latex
%%% TeX-master: t
%%% End:

\chapter{总结与展望}
\label{cha:conclusion}

\section{本文主要内容及结论}
\label{sec:conclusion}
针对入侵检测领域的相关模型没有充分利用数据集中的描述信息,以及在迁移学习过程中当数据及格式发生变化时会丢弃较多已训练的参数,本文进行以下研究:

本研究提出了一种新的通用网络分类模型,主要由通用编码器以及时序建模层两部分组成。
第二章节主要讨论了数据预处理相关部分,针对模型的特点采取了合适的数据缩放方法以及提出了一种新的对流量序列进行数据增强的方法。
在第三章提出了一种基于多模态的通用编码器,对其设计进行了详细的阐述,该编码器可灵活适应各种格式的网络流量数据,同时将数据集中数据和描述信息融入其中,将其转变为统一格式的向量,通过实验验证了多模态的通用编码器的有效性。
在第四章节对通用网络分类模型的第二部分即时序建模进行深入分析与实验,提出并验证了一个有效的网络入侵检测模型,进一步证明编码器可有效改善时序模型在不均衡数据集上的性能,避免模型过拟合。也说明基于Transformer架构的网络与第三章节提出的通用编码器搭配最为合适。
在第五章节提出了一种基于通用时序检测模型的迁移学习,证明了其相比传统的基于模型迁移学习具有较强的迁移能力,且只需更改较少的网络结构,从而保留更多的已学习参数,仅需较少数据就可实现模型对新数据集中各类别的正确分类。与近期其他入侵检测相关工作相比本文提出的方法具有一定的先进性。提出的方案具有较高的实用价值。

\section{本文主要创新点}
\label{sec:contribution}
主要的创新点是解决了网络入侵中的关键问题,即公共数据集与实际部署环境中数据集的差异会导致模型性能下降。提出的解决方案是使用通用模型以及引入数据集文本描述信息进行辅助。辅助过程中使用BERT作为句子编码器,同时引入预计算过程,在保证编码器提取信息能力的同时降低计算量。

其次为了解决前一学习过程中数据多样性低的问题,提出了一种新的基于标签数据增强方法。该方法在对数据进行变换时,保持了网络流量数据之间的时序关系,提供了高质量的数据增强结果。

\section{本文存在的不足}

\begin{enumerate}

    \item 本文提出的是一个通用模型,但只在两个数据集上进行评估,所得结论有一定局限性,在更多数据集上进行测试,会更具有说服力。
    \item 机器学习除了用于防御也可用于攻击,评估模型对于对抗样本的鲁棒性也很重要,本文中的实验并未涉模型对对抗性样本的防御能力研究。
    \item  在模型结构中,通用编码器会对网络流量数据特征进行升维操作,会带来较大的计算量。
\end{enumerate}

\section{展望}
本文提出的模型还可进一步改进,可通过以下方法:
\begin{enumerate}
    \item 在通用编码器中引入降维操作,可使用传统方法也可以使用Transformer中的注意力机制实现。
    \item 可以开展更多的实验,评估本文提出的模型在计算效率等其他方面的表现。
    \item 现在已经有性能更好的语言模型如Llama 2,可使用这些大语言模型生成更合适的位置编码或者。
    \item 在模型的分类部分引入语义描述信息,有可能进一步提升模型对网络流量,特别是少样本流量的识别准确率,甚至还可能实现零样本分类。

\end{enumerate}



\label{sec:futurework}

